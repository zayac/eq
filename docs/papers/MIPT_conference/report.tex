\documentclass[a4paper]{llncs}
\usepackage[utf8]{inputenc}
\usepackage[russian]{babel}
\usepackage{graphicx}
\usepackage{listings}
\usepackage{amsmath}
\usepackage{qtree}
\usepackage{eqcode}

\title{Язык программирования Eq}
\institute{}
\author{}
\begin{document}
\maketitle
\section*{Введение}
В данном докладе мы представляем дизайн и концепцию языка Eq, который предназначен для решения вычислительных задач. Eq основан на простом и важном наблюдении -- любая вычислительная задача может быть переформулирована как рекуррентное соотношение над параллельными по данным операциям. Неформально это можно пояснить так: любая вычислительная задача состоит из независимых друг от друга вычислений и итеративного процесса. Традиционно в императивных языках итеративный процесс записывается в виде циклов. Мы полагаем, что это представление имеет недостатки, поэтому в языке Eq мы используем другой подход, учитывающий специфику вычислительных задач.

\section*{Структурные особенности вычислительной программы}
В данном разделе иллюстрируются особенности программ, в которых реализованы вычислительные методы. В языке Eq мы используем особый подход по отношению к подобным задачам.

В качестве примера рассмотрим классическую вычислительную задачу -- краевую задачу для уравнения теплопроводности.
\subsection*{Краевая задача для уравнения теплопроводности} 
\begin{equation}
\label{eq:heat_eq}
\frac{\partial u}{\partial t} = \mu\frac{\partial^2u}{\partial x^2}\quad (\mu > 0)\  \textrm{при}\  0<t<T,\  0 < x < 1, 
\end{equation}
\begin{equation}
\label{eq:start_cond}
u(0,x) = \phi(x)
\end{equation}
\begin{equation}
\label{eq:bound_cond1}
u(t, 0) = \psi_1(t)
\end{equation}
\begin{equation}
\label{eq:bound_cond2}
u(t, 1) = \psi_2(t) 
\end{equation}
Решение задачи -- непрерывная и гладкая функция $u(t, x)$ на области $t\in(0,T), x \in (0, 1)$. При решении этой задачи на компьютере используется \emph{метод сетки}. Он заключается в том, что на область определения функции $u(t,x)$ накладывается сетка, в узлах которой ищется решение (Рис. \ref{fig:grid.png}). Таким образом решение краевой задачи для уравнения теплопроводности представимо в виде матрицы или двумерного массива.
\begin{figure}[ht]
  \centering
  \setlength{\unitlength}{7cm}
  \begin{picture}(1.2,0.9)(-0.1,-0.1)
  \put(0,0){\line(0,1){0.8}}
  \put(0,0){\line(1,0){1}} 
  \put(0.167,0){\line(0,1){0.64}}  
  \put(0.334,0){\line(0,1){0.64}}
  \put(0.501,0){\line(0,1){0.64}}  
  \put(0.668,0){\line(0,1){0.64}}  
  \put(0.835,0){\line(0,1){0.64}}   
  \put(0,0.16){\line(1,0){0.835}}   
  \put(0,0.32){\line(1,0){0.835}}      
  \put(0,0.48){\line(1,0){0.835}}  
  \put(0,0.64){\line(1,0){0.835}}    
  \put(0.336,0.32){\circle*{0.025}}
  \put(1.01,-0.06){\Large$x$}
  \put(-0.05,0.76){\Large$t$}
  \put(-0.05,-0.06){\Large$0$}  
  \put(0.325,-0.06){\Large$x_j$}  
  \put(0.828,-0.06){\Large$1$}  
  \put(-0.05,0.31){\Large$t_i$} 
  \put(-0.06,0.62){\Large$T$} 
  \put(0.35,0.35){\Large$u^i_j$} 
  \end{picture}
  \caption{Сетка, заданная на области $t \in (0, T), x \in (0, 1)$, в узлах которой считаются значения функции $u(t, x)$}
  \label{fig:grid.png}
\end{figure}
\begin{figure}[ht]
  \centering
  \setlength{\unitlength}{7cm}
  \begin{picture}(1.3,0.7)(-0.1,-0.1)
  \put(0,0){\line(1,0){1}}
  \put(-0.0265,-0.0261){\LARGE $\times$}
  \put(0.96,-0.0261){\LARGE $\times$}
  \put(0.46675,-0.0261){\LARGE $\times$}
  \put(0.5,0){\line(0,1){0.5}}
  \put(0.46675,0.4739){\LARGE $\times$}
  \put(0.52,0.4739){\Large $(i+1,j)$}
  \put(0.44,-0.08){\Large $(i,j)$}
  \put(0.9,-0.08){\Large $(i,j+1)$}
  \put(-0.1,-0.08){\Large $(i,j-1)$}
  \end{picture}
  \caption{Разностная схема для вычисления краевой задачи для уравнения теплопроводности.}
  \label{fig:template.png}
\end{figure}
Значения функции считаются с использованием \emph{разностной схемы}. Разностная схема задается шаблоном, в который входят несколько узлов сетки (Рис. \ref{fig:template.png}). Мы используем четыре соседних узла, образующие перевернутую букву "Т". Значения функции в этих точках связаны соотношением, которое позволяет по трем заданным значениям восстановить четвертое.

Из начального условия \eqref{eq:start_cond} нам известны значения функции в узлах при $t=0$. Таким образом можно использовать значения функции в узлах разностой схемы на нижнем слое при $i=0$ для подсчета внутренних точек сетки при $i=1$. Приведем без доказательства выражение в общем виде для $u^{i+1}_j$ через $u^i_{j-1}, u^i_j, u^i_{j-1}$:
\begin{equation}
\label{eq:iter_process}
u^{i+1}_j  = u^i_j + \alpha(u^i_{j-1} - 2u^i_j + u^i_{j+1}),\quad \textrm{где}\ \alpha\  \textrm{-- некоторая константа.}
\end{equation}
После вычисления узлов $u^1_j$ возможно посчитать значения в узлах $u^2_j$, используя полученные  данные. Данный процесс прекращается, когда посчитаны значения функции в узлах самого верхнего слоя. После этого можно гарантировать, что значения во всех узлах сетки посчитаны, и эти значения \emph{апроксимируют} поведение искомой функции на всей области определения.

Основываясь на приведенном решении заметим, что
\begin{itemize}
	\item \textbf{Вычисления на каждом этапе производятся независимо.} Пусть $j$ принимает целые значения $[0, n-1]$, где $n$ -- количество узлов в каждом слое. Тогда имеется $n-2$ независимых друг от друга операций.
	\item \textbf{Вычисления представляют собой итеративный процесс по индексу $i$}. Из выражения \eqref{eq:iter_process} следует, что $u_i^j$ при фиксированном $i$ могут быть вычислены независимо, а $u^i$ зависит только от $u^{i-1}$. Так формируется итеративны процесс. Он прекращается, когда найдены значения в узлах $u(T, x),\ 0<x<1$, что гарантирует наличие посчитанных значений во всех слоях. Эти значения апроксимируют искомую функцию на заданной области определения.
\end{itemize}
Для языка Eq эти принципы являются базовыми. Нами рассмотрен один пример решения вычислительной задачи на компьютере, но вышеуказанные особенности распостраняются на множество различных задач.

\section*{Концепция языка Eq}
Основная задача языка Eq -- сохранение естественной формулировки задачи. Формула \eqref{eq:iter_process} лаконично описывает процесс \eqref{eq:heat_eq}. В языке Eq эта формула сохраняет свой вид. 

Для начала покажем, как выглядит реализация итеративного процесса на языке программирования Fortran 90.
\lstinputlisting[language=Fortran,label=code:fortran,breaklines=true,numbers=left]{fortran.f90}

В императивном языке этот алгоритм будет описываться с помощью циклов. Алгоритм полностью соответсвует итеративному процессу \eqref{eq:iter_process}, но в нем отсутствуют важные детали, имеющиеся в формуле. В выражении \eqref{eq:iter_process} верхний и нижний индекс имеют разное назначение, а поэтому сознательно разделены.
Индексы $i$ и $j$ в программе пишутся рядом, и используются для адресации элементов массива. В программе не содержится информации о том, что вычисления независимы и производятся параллельно. Также отсутствует явная рекуррентная зависимость по $i$. В языке Eq мы избавляемся от традиционных циклов и вводим разделение параллельных и зависимых операций.

Возможен функциональный подход к решению задачи. Приведем пример решения задачи на языке Haskell:
\lstinputlisting[language=Haskell,label=code:haskell,breaklines=true,numbers=left]{haskell.hs}
В отличие от реализации на императивном языке здесь представлена формула \eqref{eq:iter_process} в том виде, в котором она записывается "на бумаге". Более того, код удовлетворяет вышеописанным требованиям:
\begin{itemize}
	\item $u^i$ есть функция от $u^{i-1}$, и только,
	\item итеративный процесс не зависит от $j$.
\end{itemize}
Концепция рекуррентных соотношений над параллельными операциями здесь реализована. Однако основная проблема состоит в экспоненциальном росте вызова функций (Рис. \ref{fig:func_tree}).
\begin{figure}[ht]
  \centering
\Tree [.$u^i_j$ [.$u^{i-1}_j$ $u^{i-2}_j$ $u^{i-2}_{j-1}$ $u^{i-2}_j$ $u^{i-2}_{j+1}$ ] 
[.$u^{i-1}_{j-1}$ ]
$u^{i-1}_j$ $u^{i-1}_{j+1}$
 ]
  \caption{Фрагмент дерева вызовов функций для формулы \eqref{eq:iter_process}.}
  \label{fig:func_tree}
\end{figure}

Из-за того, что посчитанные значения не запоминаются, их на каждом этапе приходится вычислять заново. Это ведет к экспоненциальному возрастанию вычислительной сложности и стека.

В языке Eq объединены императивный и функциональный подходы. Решение задачи выглядит так:
\begin{figure}[ht]
  \centering
$$
  u^{[0]} =
  \begin{tmatrix}
    0.84 \lend
    0.91 \lend
    0.14 \lend
    -0.76 \lend
    -0.96 \lend
  \end{tmatrix} $$
\end{figure}
\begin{figure}[ht]
  \centering
$$  u^{[i]}_j = 
  \begin{cases}
    \phi_i & j = 0 \lend
    \psi_i & j = 4 \lend
  u^{[i-1]}_j + \alpha \cdot (u^{[i-1]}_{j-1} + 2 \cdot u^{[i-1]}_j + u^{[i-1]}_{j+1})
  & 1 \leq j \leq 3 \lend
  \end{cases}
$$
\end{figure}

В качестве синтаксиса используется стандарт для верстки научных публикаций -- \LaTeX. Это позволяет записать в программе формулу \eqref{eq:iter_process} в неизменном виде. В Eq циклы не используются вообще. Вместо этого мы вводим понятие \emph{верхнего индекса}. Формула, в которой используется верхний индекс, должна быть записана таким образом, чтобы выражение с индексом, стоящее в левой части формулы, выражалось через выражение в правой части. Этого достаточно для представления итеративной последовательности. Зависимость будет рекуррентной, если значение функции выражается через собственные значения предыдущих итераций.
Нижний индекс используется для обращения к элементам массива. Необходимо явно задать выражение для каждого из элементов. Тогда вычисления значений элементов могут выполняться независимо.

В отличие от функционального подхода экспоненциальный рост вызова функций невозможен в случае с Eq, потому что вычисляется значение элемента массива, а не функции. В этом случае посчитанные значения сохраняются и повторного вычисления не происходит.

Описанный подход позволяет избежать недостатков, которые имеют императивные и функциональные языки. Он обобщается на большой класс вычислительных задач для которых способ записи решения \eqref{eq:iter_process} естественен. 

\section*{Выводы}
В языке Eq мы объединяем два подхода к программированию: функциональный и императивный с целью оптимального решения вычислительных задач. В рамках этой идеи мы вводим разделение индексов для массивов данных. Верхний индекс существует в единственном экземпляре и описывает итеративный процесс над массивом. Обращение к элементам массива и запись в них выполняется с помощью нижних индексов. Количество нижних индексов ограничено только размерностью самого массива. Операции над элементами массива или подмассивами выполняются параллельно. Независимое выполнение осуществимо, так как в случае каких-либо зависимостей, они решаются построением итеративного процесса с помощью операций над верхним индексом.

В основе синтаксиса для языка лежит стандарт для текстового описания \LaTeX. Это позволяет разделять индексы естественным образом. Более того, мы получаем возможность использовать явно многие математические обозначения в самом тексте программы.
\end{document}

