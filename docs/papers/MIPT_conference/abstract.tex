% Copyright (C) 2011 Moscow Institute of Physics and Technology

\documentclass[a4paper]{llncs}

\usepackage[utf8]{inputenc}
\usepackage[russian]{babel}
\renewcommand\andname{и}

\author{Зайченков П.О.\inst{1} \and Шинкаров А.Ю.\inst{2}}
\date{\today}
\title{Язык программирования Eq}
\institute{
  Московский физико-технический институт,
  Кафедра информатики и вычислительной техники
\and
  University of Hertfordshire,
  Hatfield, Hertfordshire,
  AL10 9AB, United Kingdom
}

\begin{document}

\maketitle

Решение большинства естественнонаучных проблем сопряжено с огромным количеством
вычислений, основным языком программирования для которых является Фортран. На то
имеется ряд причин: во-первых, обратная совместимость -- метеорологические
программы насчитывают миллионы строк кода, а первые версии появились в тот
момент, когда Фортран был одним из самых передовых языков программирования;
во-вторых, производительность -- такие языки, как MatLab, Python и Java,
предоставляют высокий уровень абстракций, но не могут сравниться в скорости
счета с языками низкого уровня; и, наконец, сложность разработки -- Фортран
оказался удачным компромиссом между скоростью разработки и скоростью исполнения
программы. На сегодняшний день тенденции в производстве компьютеров сменились с
гонки за тактовой частотой на увеличение количества ядер процессора. Именно этим
фактом обусловлен рост интереса к параллельному программированию. Немаловажную
роль играет появление на рынке графических ускорителей GPGPU, предоставляющих
еще б\'{о}льшие возможности для увеличения скорости программ, однако требующие
серьезного изменения парадигмы программирования. Для того, чтобы выполнить
программу параллельно, необходимо выявить участки кода без временных
зависимостей, то есть, набор операций, который можно исполнить в произвольном
порядке. Сделать это можно двумя способами: либо указать на такие участки явным
образом (такой подход, к примеру, используется в библиотеках MPI и OpenMP), либо
предоставить возможность компилятору  посредством некоторого анализа, выявить их
самостоятельно. В первом случае, программист имеет больше контроля над
исполнением, но сложность разработки и отладки такой программы увеличивается на
порядок. Во втором случае, возрастает сложность компилятора, однако уменьшается
время, необходимое на разработку.

Авторы представляют язык программирования Eq, который использует комбинацию двух
подходов и с помощью своего синтаксиса значительно упрощает анализ
параллельности кода. В нем определены две принципиально важных конструкции:
параллельное и рекуррентное выражение. Рекуррентное выражение -- это аналог нити
исполнения в операционной системе, когда одно действие должно строго следовать
за другим. Внутри параллельного выражения все действия могут быть исполнены в
любом порядке. Мы утверждаем, что двух этих конструкций, ветвления и атомарных
операций достаточно, чтобы записать любую программу на Фортране и принять
оптимальное решение о параллельности того или иного участка кода.

В качестве примера рассмотрим гнездо циклов в некоторой программе на Фортране.
Предположим, что некоторая часть операций внутри этих циклов может быть
исполнена параллельно, а значит, может быть записана с помощью параллельного
выражения. В таком случае все оставшиеся операции группируются в одно
рекуррентное выражение, где формула рекуррентости заменит шаг изменения
счетчиков внутри циклов.

Основой синтаксиса для Eq является текстовый процессор \LaTeX, являющийся
стандартом для верстки научных публикаций. В итоге программа, записанная на Eq,
понимается текстовым процессором и имеет стандартный графический оттиск; с
другой стороны, эта же программа компилируется на большинство современных
архитектур. Язык Eq позволяет разделить усилия ученого, заинтересованного
исключительно в результате определенных вычислений, и программиста,
разрабатывающего компилятор, соответствующий ситуации на современном рынке
компьютеров.  
\end{document}
