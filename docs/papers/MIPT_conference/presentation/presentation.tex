\documentclass{beamer}
\usepackage[utf8]{inputenc}
\usepackage[russian]{babel}
\usepackage{caption}
\usepackage{graphicx}
\usepackage{listings}
\usepackage{qtree}
\usepackage{tikz} 
\usepackage{amsmath, amsthm, amssymb}

\usetheme{progressbar}
\usefonttheme{professionalfonts}
\progressbaroptions{
  headline=sections,
  imagename=eq.png
}

\title {Язык программирования Eq}
\subtitle{54-я научная конференция МФТИ}
\author{Зайченков П.О.}
\date{26 ноября, 2011}
\institute{
  Московский физико-технический институт
}
\begin{document}

\maketitle

\begin{frame}
  \frametitle{Agenda}
  \tableofcontents
\end{frame}

\section{Структура программы}
\subsection*{Уравнение теплопроводности}
\begin{frame}
  \frametitle{Структура программы}
  \begin{block}{Краевая задача для уравнения теплопроводности} 
  \begin{equation}
    \label{eq:heat_eq}
    \frac{\partial u}{\partial t} = \mu\frac{\partial^2u}{\partial x^2}\quad (\mu
    > 0)\  \textrm{при}\  0<t<T,\  0 < x < 1, 
  \end{equation}
\begin{equation}
\label{eq:start_cond}
u(0,x) = \phi(x)
\end{equation}
\begin{equation}
\label{eq:bound_cond1}
u(t, 0) = \psi_1(t)
\end{equation}
\begin{equation}
\label{eq:bound_cond2}
u(t, 1) = \psi_2(t) 
\end{equation}
  \end{block}
\end{frame}

\begin{frame}
  \begin{block}{Метод сетки}
\begin{figure}[ht]
  \centering
  \setlength{\unitlength}{5cm}
  \begin{picture}(1.2,0.9)(-0.1,-0.1)
  \put(0,0){\line(0,1){0.8}}
  \put(0,0){\line(1,0){1}} 
  \put(0.167,0){\line(0,1){0.64}}  
  \put(0.334,0){\line(0,1){0.64}}
  \put(0.501,0){\line(0,1){0.64}}  
  \put(0.668,0){\line(0,1){0.64}}  
  \put(0.835,0){\line(0,1){0.64}}   
  \put(0,0.16){\line(1,0){0.835}}   
  \put(0,0.32){\line(1,0){0.835}}      
  \put(0,0.48){\line(1,0){0.835}}  
  \put(0,0.64){\line(1,0){0.835}}    
  \put(0.336,0.32){\circle*{0.025}}
  \put(1.01,-0.06){\large$x$}
  \put(-0.05,0.76){\large$t$}
  \put(-0.05,-0.06){\large$0$}  
  \put(0.325,-0.06){\large$x_j$}  
  \put(0.828,-0.06){\large$1$}  
  \put(-0.05,0.31){\large$t_i$} 
  \put(-0.06,0.62){\large$T$} 
  \put(0.35,0.35){\large$u^i_j$} 
  \end{picture}
  $$t \in (0, T), x \in (0, 1)$$
  $$ u(t,x):\quad u(0,x) = \phi(x), u(t, 0) = \psi_1(t), u(t, 1) = \psi_2(t) $$
  \end{figure}
  \end{block}
\end{frame}

\begin{frame}
  \begin{block}{Разностная схема}
  \centering
  \setlength{\unitlength}{5cm}
  \begin{picture}(1.3,0.7)(-0.1,-0.1)
  \put(0,0){\line(1,0){1}}
  \put(-0.0265,-0.0261){\Large $\times$}
  \put(0.96,-0.0261){\Large $\times$}
  \put(0.46675,-0.0261){\Large $\times$}
  \put(0.5,0){\line(0,1){0.5}}
  \put(0.46675,0.4739){\Large $\times$}
  \put(0.52,0.4739){\large $(i+1,j)$}
  \put(0.44,-0.08){\large $(i,j)$}
  \put(0.9,-0.08){\large $(i,j+1)$}
  \put(-0.1,-0.08){\large $(i,j-1)$}
  \end{picture}
  \label{fig:template.png}
  $$u^{i+1}_j  = u^i_j + \alpha(u^i_{j-1} - 2u^i_j + u^i_{j+1})$$
  \end{block}
\end{frame}

\begin{frame}
	\begin{exampleblock}{\ }
	  $$u^{i+1}_j  = u^i_j + \alpha(u^i_{j-1} - 2u^i_j + u^i_{j+1})$$
	\begin{itemize}
	\item Вычисления на каждом этапе производятся независимо.  
	\item Вычисления представляют собой итеративный процесс по индексу $i$.
\end{itemize}
	\end{exampleblock}
\end{frame}

\section{Концепция языка Eq}
\begin{frame}
\frametitle{Концепция языка Eq}
$$u^{i+1}_j  = u^i_j + \alpha(u^i_{j-1} - 2u^i_j + u^i_{j+1})$$
	\scriptsize{
	\lstinputlisting[language=Fortran,frame=single,
	label=code:fortran,breaklines=true,numbers=left,caption={Язык C}]{c.c}
	}
\end{frame}

\begin{frame}
$$u^{i+1}_j  = u^i_j + \alpha(u^i_{j-1} - 2u^i_j + u^i_{j+1})$$
	\begin{figure}[ht]
	\scriptsize{
	\lstinputlisting[language=Haskell,frame=single,
	label=code:haskell,breaklines=true,numbers=left,caption={Haskell}]{haskell.hs}
	}
	\end{figure}
\end{frame}

\begin{frame}
	$$u^{i+1}_j  = u^i_j + \alpha(u^i_{j-1} - 2u^i_j + u^i_{j+1})$$
	\begin{figure}[ht]
	  \centering
\Tree [.$u^i_j$ [.$u^{i-1}_j$ $u^{i-2}_j$ $u^{i-2}_{j-1}$ $u^{i-2}_j$
$u^{i-2}_{j+1}$ ] [.$u^{i-1}_{j-1}$ ]
$u^{i-1}_j$ $u^{i-1}_{j+1}$
 ]
	\end{figure}
\end{frame}

\begin{frame}

\begin{exampleblock}{$u^{i+1}_j  = u^i_j + \alpha(u^i_{j-1} - 2u^i_j +
u^i_{j+1})$} $$
  u^{[0]} =
  \begin{pmatrix}
    0.84 \\
    0.91 \\
    0.14 \\
    -0.76 \\
    -0.96 \\
  \end{pmatrix} $$
$$  u^{[i]}_j = 
  \begin{cases}
    \phi_i & j = 0 \\
    \psi_i & j = 4 \\
  u^{[i-1]}_j + \alpha \cdot (u^{[i-1]}_{j-1} + 2 \cdot u^{[i-1]}_j +
  u^{[i-1]}_{j+1}) & 1 \leq j \leq 3 \\
  \end{cases}
$$
\end{exampleblock}
\end{frame}

\begin{frame}

	\begin{figure}[ht]
	\scriptsize{
	\lstinputlisting[frame=single,
	label=code:haskell,breaklines=true,numbers=left,caption={\LaTeX}]{latex.tex}
	}
	\end{figure}
\end{frame}

\section{Выводы}
\begin{frame}
\frametitle{Выводы}
\begin{exampleblock}{\ }
	\begin{itemize}
		\item Объединение функционального и императивного подходов.
		\item Параллельное исполнение независимых участков.
		\item Объединение в рекуррентные соотношения.
		\item \LaTeX в качестве синтаксиса
	\end{itemize}
\end{exampleblock}
\end{frame}
\begin{frame}
	\centering \underline{University of Hertfordshire} \\
	\centering \underline{Compiler Technology and Computer Architecture Group} \\	
	
	\centering http://github.com/zayac/EqCode \\
	\centering zaichenkov@gmail.com \\
\end{frame}
\end{document}
