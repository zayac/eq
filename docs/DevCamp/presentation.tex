\documentclass[xcolor=dvipsnames,mathserif,professionalfont,12pt]{beamer} 
\usecolortheme[named=RoyalBlue]{structure} 
\useoutertheme{infolines} 
\usetheme[height=7mm]{Rochester} 
\setbeamertemplate{items}[ball] 
\setbeamertemplate{blocks}[rounded][shadow=true] 
\setbeamertemplate{navigation symbols}{}
%\usefonttheme{serif}
\usefonttheme{professionalfonts}

\usepackage[croatian]{babel}
\usepackage[utf8x]{inputenc}
\usepackage{txfonts}
\usepackage{listings}
\usepackage{color}
\usepackage{textcomp}

\definecolor{listinggray}{gray}{0.9}
\definecolor{lbcolor}{rgb}{0.9,0.9,0.9}
\lstset{
	tabsize=2,
	rulecolor=,
	language=,
  basicstyle=\scriptsize,
  upquote=true,
  columns=fixed,
  showstringspaces=false,
  extendedchars=true,
  breaklines=true,
  prebreak = \raisebox{0ex}[0ex][0ex]{\ensuremath{\hookleftarrow}},
  identifierstyle=\ttfamily,
  keywordstyle=\color[rgb]{0,0,1},
  commentstyle=\color[rgb]{0.133,0.545,0.133},
  stringstyle=\color[rgb]{0.627,0.126,0.941},
}

\title[Eq Programming Language]{Eq Programming Language}
\author[Pavel Zaichenkov]{Pavel Zaichenkov}
\institute[UoH]{University of Hertfordshire}
\date{July 26, 2011}
\begin{document}
\begin{frame}
  \titlepage
\end{frame}

\begin{frame}{Agenda}
  \tableofcontents
\end{frame}

\begin{frame}[fragile]
  \frametitle{A language concept}
  In computational mathematics or physics all operations can be separated into
  two types. \\
  \textbf{Data parallel operation} doesn't depend on previous iterations. It
  deals with independent data. In this way, all computational processes can be
  run separately.\\ 
  \textbf{Recurrent depended operations} can't be run separately. Iterations
  have to be run in a queue, as an every next operation is going to use the result
  of the previous one. \\ 
\quad \\
  In mathematics there is a widely used notation, which seems quite easy to
  understand because of formulas, equations and mathematical designations. 
\end{frame}

\begin{frame}[fragile]
  \frametitle{A language concept}
  \begin{block}{Data parallel operation}
%If the data is independent, it's possible to run computational tasks
%in several processes. \\
  \begin{align*}
f_0=f(x_0, x_1,\ ...\ x_n) \\
f_1=f(x_0, x_1,\ ...\ x_n) \\
 ... \quad \ \  \\
f_i=f(x_0, x_1,\ ...\ x_n) \\
\text{where}\  \forall i\ x_i\  \text{is data }  \\
    \end{align*}
  \end{block}
\end{frame}

\begin{frame}[fragile]
  \frametitle{A language concept}
  \begin{block}{Recurrent depended operation}
%Recurrent depended data can be represented as an infinite list or a stream. It
%allows us to filter a necessary range of data. Required data will be calculated
%automatically. This is a functional language concept.\\ 
\begin{align*}
f_i=f(f_j, f_{j+1},\ ...\ f_{i-1}, x_1,\ ...\ x_n) \\
\text{where}\ \forall i\ x_i\  \text{is data } \  \text{and}\  j \leq i-1 \\
\end{align*}
  \end{block}
\end{frame}

\begin{frame}{The structure of a compiler}
  \begin{enumerate}
    \item \textbf{\LaTeX\  front-end}. It's possible to write a program using
    \LaTeX\  syntax. This allows to use any existing \LaTeX\  tools (compile to
    pdf, ps, html, etc...). \\
    \item \textbf{EqCode compiler}. We are going to write a compiler which will
    be able to compile an existing \LaTeX\  code into any chosen back-end
    language.
    \item \textbf{Custom back-end (SaC, S-Net, C, ...)}. It's possible to
    create a code-generator into any language we want to deal with. We are
    going to support SaC as it has a relevant data parallelism and recurrent
    dependency support.
  \end{enumerate}
\end{frame}

\begin{frame}[fragile]
  \frametitle{Fibonacci numbers}
  \begin{columns}[t]
    \column{.4\textwidth}
      \begin{block}{Wikipedia}
        \begin{align*}
F_0 &= 0  \\
F_1 &= 1  \\ 
F_i &= F_{i-1} + F_{i-2}
        \end{align*}
      \end{block}
    \column{.5\textwidth}
      \begin{block}{C / SaC}
        \begin{lstlisting}
int f(int n)
{
  if ((n == 0) || (n == 1))
    return n;
  return f(n - 1) + f(n - 2);
}
        \end{lstlisting}
    \end{block}
  \end{columns}
\end{frame}

%\begin{frame}{Fibonacci numbers}
%      \begin{block}{M
%        \begin{align*}
%F_0 &= 0  \\
%F_1 &= 1  \\ 
%F_i &= F_{i-1} + F_{i-2}
%        \end{align*}
%      \end{block}
%      \begin{itemize}
%        \item Familiar
%        \item Short
%        \item Easy to understand
%        \item Customizable backend
%      \end{itemize}
%\end{frame}
 
\begin{frame}[fragile]
  \frametitle{N-body problem}
  \begin{block}{Wikipedia}
    \begin{align*}
m_j	\ddot{\mathbf{q}}_j	= G \sum\limits_{k\neq j }  \frac{m_j
m_k(\mathbf{q}_k-\mathbf{q}_j)}{|\mathbf{q}_k-\mathbf{q}_j|^3}, j=1,\ldots,n \\
    \end{align*}
  \end{block}
\end{frame}

\begin{frame}[fragile]
  \frametitle{N-body problem}
\begin{block}{C\  (debian shootout)}
    \begin{lstlisting}
void advance(int nbodies, struct planet * bodies, double dt)
{
  int i, j;
  for (i = 0; i < nbodies; i++) {
    struct planet * b = &(bodies[i]);
    for (j = i + 1; j < nbodies; j++) {
      struct planet * b2 = &(bodies[j]);
      double dx = b->x - b2->x; double dy = b->y - b2->y; double dz = b->z - b2->z;
      double distance = sqrt(dx * dx + dy * dy + dz * dz);
      double mag = dt / (distance * distance * distance);
      b->vx -= dx * b2->mass * mag; b->vy -= dy * b2->mass * mag; b->vz -= dz * b2->mass * mag;
      b2->vx += dx * b->mass * mag; b2->vy += dy * b->mass * mag; b2->vz += dz * b->mass * mag;
    }
  }
  for (i = 0; i < nbodies; i++) {
    struct planet * b = &(bodies[i]);
    b->x += dt * b->vx; b->y += dt * b->vy; b->z += dt * b->vz;
  }
}
    \end{lstlisting}
  \end{block}
\end{frame}

\begin{frame}[fragile]
  \frametitle{N-body problem}
\begin{block}{EqCode}
  \begin{align*}
& advance(p, v, m, dt):\ \mathbb{R}^2_{5,3}, \mathbb{R}^2_{5,3}, \mathbb{R}^1_5,
\mathbb{R}\  \rightarrow\ \mathbb{R}^3 \\
& \quad  accs_{i,j}\  |\  0 \leq i \leq 4 \land 0 \leq j \leq 4 =
        \begin{cases}
            \dfrac{(p_j-p_i) \cdot m_j}{\rho(p_i,p_j)^3} & j < i \\
 0 & \text{otherwise}
       \end{cases} \\
& \quad accs_{i,j}\  |\  j > i = - accs_{j,i} \\
& \quad   a_{i,j} = \sum \limits_k accs_{i,k,j} \\
& \quad   v = v + a \cdot dt \\
& \quad   p = p + v \cdot dt \\
& \quad  \text{return}(p, v) \\
  \end{align*}
\end{block}
\end{frame}

\section{Introduction}
\subsection{A language concept}
\section{EqCode description}
\subsection{The structure of a compiler}
\subsection{Fibonacci numbers\  (example)}
\subsection{N-body problem\ (example)}
\subsection{Syntax and language constructions}
\section{Problems and restrictions}

\begin{frame}[fragile]
  \frametitle{Recurrent operations support}
  \begin{columns}[t]
    \column{.4\textwidth}
    \begin{block}{EqCode}
      \begin{align*}
& f(n):\  \mathbb{Z}\  \rightarrow\  \mathbb{Z} \\
& \quad F^{[0]} = 0 \\
& \quad F^{[1]} = 1 \\
& \quad F^{[i]} = F^{[i-1]} + F^{[i-2]} \\
& \quad {\textbf {return}}({\textbf {filter}}(F^{[i]}\  |\  i = n))
      \end{align*}
    \end{block}
    \column{.6\textwidth}
      \begin{block}{\LaTeX}
      \begin{lstlisting}
\begin{eqcode}{f}{n}{\type{Z}}{\type{Z}}
  F^{[0]} = 0 \lend
  F^{[1]} = 1 \lend
  F^{[i]} = F^{[i-1]} + F^{[i-2]} \lend
  \return{\filter{F^{[i]}\  |\  i = n}}
\end{eqcode}
      \end{lstlisting}
      \end{block}
  \end{columns}
\end{frame}

\begin{frame}[fragile]
  \frametitle{Parallelized operations support}
    \begin{block}{EqCode}
      \begin{align*}
a_{i,j}\  |\  0 \leq i < 5 \land 2 \leq j < 6 =
\begin{cases}
42 & 0 \leq i < 2 \land 0 \leq j < 3 \\
0 & \text{otherwise}
\end{cases}
      \end{align*}
    \end{block}
      \begin{block}{\LaTeX}
      \begin{lstlisting}
a_{i,j}\  |\  0 \leq i < 5 \land 2 \leq j < 6 = 
  \begin{cases}
    42 & 0 \leq i < 2 \land 0 \leq j < 3 \lend
    0 \otherwise
  \end{cases}
      \end{lstlisting}
      \end{block}
      \begin{block}{SaC}
      \begin{lstlisting}
a = with {
    ([0,2] <= [i,j] < [5,6] ) : 42;
  } : genarray([5,6], 0);
      \end{lstlisting}
      \end{block}
\end{frame}

%\begin{frame}[fragile]
%  \frametitle{Vector}
%    \begin{block}{EqCode}
%      \begin{align*}
%      w \in \mathbb{Z}^1 \\
%      w = \left(
%        \begin{array}{c}
%        1 \\
%        2 \\
%        3 \\
%      \end{array}
%      \right)
%      \end{align*}
%    \end{block}
%      \begin{block}{\LaTeX}
%      \begin{lstlisting}
%      w \in \type{Z}^1 \lend
%      w = \begin{vector}
%        1 \lend
%        2 \lend
%        3 \lend
%      \end{vector}
%      \end{lstlisting}
%      \end{block}
%\end{frame}

%\begin{frame}[fragile]
%  \frametitle{Matrix}
%    \begin{block}{EqCode}
%      \begin{align*}
%      w \in \mathbb{Z}^2 \\
%      w = \left(
%        \begin{array}{cc}
%        1 & 2\\
%        3 & 4 \\
%      \end{array}
%      \right)
%      \end{align*}
%    \end{block}
%      \begin{block}{\LaTeX}
%      \begin{lstlisting}
%      w \in \type{Z}^2 \lend
%      w = \begin{matrix}{cc}
%        1 & 2\lend
%        3 & 4 \lend
%      \end{matrix}
%      \end{lstlisting}
%    \end{block}
%\end{frame}

\begin{frame}{Problems and restrictions}
  \begin{itemize}
    \item \textbf{Types and type conversion issues}. We can't use just
    types that mathematicians are familiar with( natural numbers, whole
    numbers, etc.). There should be a type hierarchy to understand the
    representation of these types in the architecture.
    \item \textbf{Syntax restrictions}. The same algorithm
    can be represented In \LaTeX\ in different ways. However, a source code
    should be translated unambiguously into the target code. That's why some
    syntax restrictions are needed.
  \end{itemize}
\end{frame}


\begin{frame}
  \begin{block}{Project repository}
    http://github.com/zayac/EqCode/
  \end{block}
  \begin{block}{Contacts}
    Mail+Jabber: zaichenkov@gmail.com
  \end{block}
\begin{center} Questions? \end{center}
\end{frame}
\end{document}
