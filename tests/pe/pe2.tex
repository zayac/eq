
\title{An algorithm example in \LaTeX\  and Eq}
\author{
        Pavel Zaichenkov \\
        Compiler Technology and Computer Architecture Group\\
	University of Hertfordshire\\
        Hatfield, \underline{United Kingdom}
}
\date{\today}

\documentclass[12pt]{article}

\usepackage{eqcode}
\begin{document}
\maketitle

\section{Formulation of a problem}
Each new term in the Fibonacci sequence is generated by adding the previous two terms. By starting with 1 and 2, the first 10 terms will be:

\begin{align*}
1, 2, 3, 5, 8, 13, 21, 34, 55, 89, ...
\end{align*}

\emph{By considering the terms in the Fibonacci sequence whose values do not
exceed four million, find the sum of the even-valued terms.}

\section{Algorithm}
\begin{eqcode}{main}{\ }{\ }{\type{Z}}
  a^{[0]} \gets 1,
  b^{[0]} \gets 2,
  x^{[0]} \gets 0 \lend
  c^{[i]} \gets a^{[i-1]} + b^{[i-1]},
  a^{[i]} \gets b^{[i-1]},
  b^{[i]} \gets c^{[i-1]} \lend
  \qif {a^{[i]} \mod 2 = 0}
    x^{[i]} \gets x^{[i-1]} + a^{[i]} \lend
  \qendif
  \return {\filter {x^{[i]} | a : a^{[i]} > 4 \cdot 10^6 }} \lend

\end{eqcode}

\section{Answer}
  The answer is \bf{4613732}

\end{document}
