
\title{An algorithm example in \LaTeX\  and Eq}
\author{
        Pavel Zaichenkov \\
        Compiler Technology and Computer Architecture Group\\
	University of Hertfordshire\\
        Hatfield, \underline{United Kingdom}
}
\date{\today}

\documentclass[12pt]{article}

\usepackage{eqcode}
\begin{document}
\maketitle

\section{Formulation of a problem}
If we list all the natural numbers below 10 that are multiples of 3 or 5, we
get 3, 5, 6 and 9. The sum of these multiples is 23. \\
\emph{Find the sum of all the multiples of 3 or 5 below 1000.}

\section{Algorithm}
\begin{eqcode}{main}{\ }{\ }{\type{Z}}
  \return{\call{\gamma}{3, 999} + \call{\gamma}{5, 999} - \call{\gamma}{15,
  999}} \lend
\end{eqcode}

\begin{eqcode}{floor}{a}{\type{R}}{\type{Z}}
\end{eqcode}

\begin{eqcode}{\gamma}{n, m}{\type{Z}, \type{Z}}{\type{Z}}
  \match{\lfloor \expr \rfloor}{\call{floor}{\expr {1}}}
  \match{\left \lfloor \expr \right \rfloor}{\lfloor \expr{1} \rfloor}
  f \gets \left \lfloor  \dfrac{m}{n} \right \rfloor \lend
  \return{n + f \cdot \frac{f+1}{2}} \lend
\end{eqcode}

\section{Answer}
  The answer is \bf{233168}

\end{document}
